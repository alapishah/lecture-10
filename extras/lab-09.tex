\documentclass{tufte-handout}

\usepackage{xcolor}

% set image attributes:
\usepackage{graphicx}
\graphicspath{ {images/} }

% set hyperlink attributes
\hypersetup{colorlinks}

% ======================================================

% define the title
\title{SOC 4650/5650: Lab-09 - Health Insurance Rates by County}
\author{Christopher Prener, Ph.D.}
\date{Spring 2019}

% ======================================================

\begin{document}

% ======================================================

\maketitle % generates the title

% ======================================================

\vspace{5mm}
\section{Directions}
Using data accessed from the \texttt{lecture-10} repository, \texttt{USHealth}, and the \texttt{US\_BOUNDARY\_Counties} shapefile, create the maps below related to health insurance rates by county for all fifty states. Your entire project folder system, including data and RMarkdown output, should be uploaded to GitHub by \textbf{Monday, March 25\textsuperscript{th}} at 4:15pm.

\vspace{5mm}
\section{Analysis Development}
The goal of this section is to create a self contained project directory with all of the data, code, map documents, results, and documentation a project needs. Please ensure \textbf{all} required elements are present. You will need to add the \texttt{US\_BOUNDARY\_Counties} shapefile to your project. You will also need the \texttt{US\_HEALTH\_noIns.csv} and \texttt{stateCapitals.csv} files.

\vspace{5mm}
\section{Part 1: Data Preparation}
The goal of this section is to produce two shapefiles from the raw data provided. These shapefiles should illustrate the percent of individuals lacking health insurance by county in the United States as well as the point location of state capitals. 

\begin{enumerate}
\item Using \texttt{R}, complete the following steps:
\begin{enumerate}
\item Import the \texttt{stateCapitals.csv} data and project it using the included \texttt{x,y} coordinate data. Check your projection using \texttt{mapview} before exporting the data. The resulting data should be exported as a shapefile with the \texttt{NAD 1983} geographic coordinate system applied. The shapefile should be saved to a subfolder of \texttt{data/} named \texttt{cleanData/} in your lab's folder hierarchy.
\item Import both the county boundary data and the health insurance data, and complete a table join to combine both data sets. 
\item There are some values of \texttt{-1} in the insurance data. Those are ``missing'' counties that the CDC does not provide insurance rate estimates for. To remove them, subset your observations so that you only have observations remaining where the variable \texttt{noIns} is greater than or equal to \texttt{0}.
\item The resulting data should be exported as a shapefile with the \texttt{NAD 1983} geographic coordinate system applied. The shapefile should be saved to a subfolder of \texttt{data/} named \texttt{cleanData/} in your lab's folder hierarchy.
\end{enumerate}
\end{enumerate}

\vspace{5mm}
\section{Part 2: Mapping Health Insurance Data for the Contiguous United States}
The goal of this section is to produce a stand-alone map of the contiguous United States (i.e. the ``lower 48'' states) that shows the percent of individuals lacking health insurance by county.

\begin{enumerate}
\setcounter{enumi}{1}
\item In a new map document, add both the county health insurance estimate and state capital shapefiles created in the previous section. Symbolize the state capitals using a symbol that includes a star, and keep them on top of the other data. Also add the state boundary data from the \texttt{US\_BOUNDARY\_Counties} shapefile to your map.
\item Select a projected coordinate system for this map that is appropriate for mapping the contiguous United States (i.e. the ``lower 48'') - either the Albers or Lambert projected coordinate systems.
\item Create a thematic choropleth map for that shows variation in the number of individuals without health insurance. Make sure to use a Color Brewer palette as well as Jenks Natural Breaks with 5 data classes for your symbology.
\item Overlay the state boundaries (symbolized with a hollow fill) to make it easier to identify states that have not seen large decreases in the uninsured population since the introduction of the Affordable Care Act.
\item Export the map image as a \texttt{pdf} at \texttt{300dpi},
\end{enumerate}

\vspace{5mm}
\section{Part 3: Mapping Health Insurance Data for Alaska}
The goal of this section is to produce a stand-alone map of Alaska that shows the percent of individuals lacking health insurance by borough (the equivalent of counties in Alaska).

\begin{enumerate}
\setcounter{enumi}{6}
\item In a new data frame, copy the data from the previous section and change the extent of the map so that it shows only Alaska. Notice the counties that are white - these are the counties that had missing data that we managed in our query in the first section.
\item Change the projected coordinate system of this second data frame so that it is appropriate for mapping Alaska - the Albers state system for Alaska.
\item Re-position your map image to accommodate any changes to the shape of your data.
\item Export the map image as a \texttt{pdf} at \texttt{300dpi}.
\end{enumerate}

\vspace{5mm}
\section{Part 4: Mapping Health Insurance Data for Hawaii}
The goal of this section is to produce a stand-alone map of Hawaii that shows the percent of individuals lacking health insurance by county.

\begin{enumerate}
\setcounter{enumi}{10}
\item In a new data frame, copy the data from the previous section and change the extent of the map so that it shows only Hawaii.
\item Change the projected coordinate system of this third data frame so that it is appropriate for mapping Hawaii - the Albers state system for Hawaii.
\item Re-position your map image to accommodate any changes to the shape of your data.
\item Export the map image as a \texttt{pdf} at \texttt{300dpi}.
\end{enumerate}

% ======================================================
\end{document}